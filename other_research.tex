\documentclass[12pt, letterpaper]{article}
\usepackage{fancyhdr}
\usepackage[margin=1.87cm]{geometry}%rounded up from 1.87, just to be safe...
%\usepackage{parskip}
\usepackage{times} %make sure that the times new roman is used

\usepackage{xspace}
\newcommand{\etal}{\hbox{\emph{et al.}}\xspace}
\newcommand{\eg}{\hbox{\emph{e.g.,}}\xspace}
\newcommand{\ie}{\hbox{\emph{i.e.}}\xspace}

\usepackage{url}

\makeatletter
\renewcommand{\section}{\@startsection
{section}%		     % the name
{1}%			     % the level
{0mm}%			     % the indent
%{-\baselineskip}%	      % the before skip
{.75mm}
{.03\baselineskip}%	     % the after skip
{\normalfont\large\bf} % the style
}

\begin{document}

\fancyhead{}
%\fancyhf{}
\pagestyle{fancy}
\rhead{-- 309207-- \textbf{Rigby, Shihab, Shang}} %This puts your name at the top right, outside the margin
\renewcommand{\headrulewidth}{0pt}

\begin{center}
\begin{LARGE}
\noindent
\center{{\bf Relationship To Other Research Support}}
\end{LARGE}
\end{center}

\pagenumbering{gobble}

The main goal of Rigby's NSERC Discovery grant, “Contemporary Software Peer Review:
Modern practices, fault prediction, and extraction of design decisions” is to
understand the software review practices in industrial firms as opposed to open
source projects he studied in my thesis (from this grant, I have published a
paper at the top tier conference “Foundations of Software Engineering”). 
%
In the proposed grant, I will be able to use my expertise and scripts to
extract Ericsson data. A possible synergy between the grants, would be to use
peer review data in models for test prioritization. 

The goal of Rigby's FRQNT grant is to translate code elements, classes and methods,
from English into French. This grant does not overlap with the proposed grant.


\end{document}
