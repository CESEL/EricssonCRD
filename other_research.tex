\documentclass[12pt, letterpaper]{article}
\usepackage{fancyhdr}
\usepackage[margin=1.87cm]{geometry}%rounded up from 1.87, just to be safe...
%\usepackage{parskip}
\usepackage{times} %make sure that the times new roman is used

\usepackage{xspace}
\newcommand{\etal}{\hbox{\emph{et al.}}\xspace}
\newcommand{\eg}{\hbox{\emph{e.g.,}}\xspace}
\newcommand{\ie}{\hbox{\emph{i.e.}}\xspace}

\usepackage{url}

\makeatletter
\renewcommand{\section}{\@startsection
{section}%		     % the name
{1}%			     % the level
{0mm}%			     % the indent
%{-\baselineskip}%	      % the before skip
{.75mm}
{.03\baselineskip}%	     % the after skip
{\normalfont\large\bf} % the style
}

\begin{document}

\fancyhead{}
%\fancyhf{}
\pagestyle{fancy}
\rhead{-- 309207-- \textbf{Rigby, Shihab, Shang}} %This puts your name at the top right, outside the margin
\renewcommand{\headrulewidth}{0pt}

\begin{center}
\begin{LARGE}
\noindent
\center{{\bf Relationship To Other Research Support}}
\end{LARGE}
\end{center}

\pagenumbering{gobble}

Below we describe the grants held by Drs. Rigby, Shihab, and Shang.

\textbf{Dr. Rigby} holds an NSERC Discover, an FRQNT grant, and co-holds a MITACS.

The main goal of Rigby's \textbf{NSERC Discovery} grant, “Contemporary Software Peer Review:
Modern practices, fault prediction, and extraction of design decisions” is to
understand the software review practices in industrial firms as opposed to open
source projects he studied in my thesis (from this grant, I have published a
paper at the top tier conference “Foundations of Software Engineering”). 
%
In the proposed grant, I will be able to use my expertise and scripts to
extract Ericsson data. A possible synergy between the grants, would be to use
peer review data in models for test prioritization. 

The goal of Rigby's \textbf{FRQNT grant} is to translate code elements, classes and methods,
from English into French. This grant does not overlap with the proposed grant.

%%%%%%%%%%%%%%%%%%%%%%%%%%%%%%%%%%%%%%%%%%%%%%%%%%%%
\textbf{Dr. Shihab} currently holds three active grants. Through his existing funding, Dr. Shihab is supporting his current research program, which includes 4 graduate students (and an additional two student expected to start in September 2016). The three grants are:

\textbf{MITACS Accelerate:} Dr. Shihab and Dr. Rigby hold a MITACS Accelerate grant with Ericsson Ottawa (\$30,000 in the period Feb.-Sep. 2016). The MITACS Accelerate grant focuses on software analytics for test prioritization and has funded Rehman and Amar, whose work laid the ground work for the proposed project.

\textbf{NSERC Discovery:} Dr. Shihab holds an NSERC Discovery grant (\$170,000 in the period 2015-2020). The NSERC Discovery grant focuses on further advancing his research on the detection of risky software changes. The outcomes of the NSERC discovery grant can help us determine what are good indicators of risky commits.

\textbf{NSERC Engage:} Dr. Shihab holds an NSERC Engage grant (\$25,000 in the period May-October 2016). The NSERC Engage grant focuses on developing analytics for wearable technology in order to save energy and perform accurate activity detection through sensor data.
\end{itemize}

In the proposed CRD grant, I will use my expertise in software repository mining and software analytics to build models that detect and prioritize tests for risky changes.

%%%%%%%%%%%%%%%%%%%%Ian's other research grant%%%%%%%%%%%%%%%%%%%
\textbf{Dr. Shang} is currently holding a NSERC Discovery grant, a startup grant from Concordia University and a Microsoft Azure Award. Here we explain the relationship of this proposal to the above three grants:

\textbf{NSERC Discovery} grant (Awarded): Shang received a NSERC Discovery grant ``Log Intelligence: Systematically Leveraging Logs Using Development Knowledge''. The goal of this grant is to systematically help developers make logging decisions and improve log analysis, using the rich historical knowledge from software development. The Discovery grant is targeting production logs instead of testing logs in this proposal. The success of project from Shang's NSERC Discovery grant may further improve the effectiveness of techniques in the proposed grant.

\textbf{Startup Grant} (Awarded): As a new faculty member at Concordia, Shang received a start-up grant of \$50,000 for three years (from August 2015 - April 2018). The start-up funds are being used to perform studies on the modelling the performance of large software systems, which is not related to this proposal. 

\textbf{Microsoft Azure Award} (Awarded): Shang received a Microsoft Azure Award (\$20,000) in May, 2016. The award provides infrastructure support for my research titled ``Evaluating the use of cloud environment for performance testing''. The proposed research aims to study the discrepancy between performance testing results in cloud and physical environment. The proposed research leverages Microsoft Azure cloud environment to conduct performance tests and identify the discrepancy and stability of performance testing results compared to physical environment. The Microsoft Azure Award is not relevant to the work described in this proposal.

\end{document}
