\documentclass[12pt, letterpaper]{article}
\usepackage{fancyhdr}
\usepackage[margin=1.87cm]{geometry}%rounded up from 1.87, just to be safe...
%\usepackage{parskip}
\usepackage{times} %make sure that the times new roman is used

\usepackage{xspace}
\newcommand{\etal}{\hbox{\emph{et al.}}\xspace}
\newcommand{\eg}{\hbox{\emph{e.g.,}}\xspace}
\newcommand{\ie}{\hbox{\emph{i.e.}}\xspace}

\usepackage{url}

\makeatletter
\renewcommand{\section}{\@startsection
{section}%		     % the name
{1}%			     % the level
{0mm}%			     % the indent
%{-\baselineskip}%	      % the before skip
{.75mm}
{.03\baselineskip}%	     % the after skip
{\normalfont\large\bf} % the style
}

\begin{document}

\fancyhead{}
%\fancyhf{}
\pagestyle{fancy}
\rhead{Test Prioritization and Localization at Ericsson -- 309207-- \textbf{Peter C Rigby}} %This puts your name at the top right, outside the margin
\renewcommand{\headrulewidth}{0pt}

\begin{center}
\begin{LARGE}
\noindent
\center{{\bf Test Prioritization and Localization at Ericsson}}
\end{LARGE}
\end{center}

\section*{Overview of the Company-Specific Problem}

A cellular base station connects cell phones to a voice and data network via
LTE (4G) or 3G standards. The software that runs on these base stations
contains not only complex signalling logic with stringent real-time constraints,
but also must be highly reliable, providing safety critical services, such as
911 calling. At Ericsson Ottawa, testing base station software is time
consuming and involves expensive specialized hardware. For example, testers may
need to simulate cellular devices, such as when a base station is overwhelmed
by requests from cell users at a music concert. In order to maximize the return
on investment of Ericsson's testing efforts, we plan to use historical
artifacts including past code changes, bugs, peer reviews, and test runs to create
statistical models to (1) prioritize tests and (2) help developers
categorize and locate the cause of a failure. The outcome of this work will
advance the state-of-the-art in test prioritization and localization and
provide Ericsson with a more efficient test infrastructure.

\section*{Background, Research Approach, and Data}

The proposed work intersects two areas of software engineering: test
prioritization and empirical software engineering. 
%
There are three streams of regression testing research \cite{Yoo2012STVR}.
%
%: minimization, selection, and priorization\cite{Yoo2012STVR}.
%
The first, {\it minimization}, involves eliminating tests that are redundant or
of low value. In the literature, the problem has been reduced to one of code
coverage, for example, tests become redundant as the system evolves and more
than one test covers the same control flow. As a result, much of the work in
this area is algorithmic, such as transforming it into a spanning set problem
\cite{Marre2003TSE}, using divide-and-conquer strategies \cite{Chen1996IPL},
and greedy algorithms \cite{Tallam2005SENotes}. 
%
%These algorithms are based on source code coverage, and miss important
%historical information, such as whether a file is at higher risk to failure
%because it has been changed recently.
The second, {\it selection}, uses the same static analysis techniques such as
coverage \cite{Taha1989COMPSAC} and slicing \cite{Jeffrey2006COMPSAC}, but
selects tests that cover source files that are at higher risk because they have
been changed recently \cite{Rothermel1994ICSE}. 
%
The third, {\it prioritization}, orders tests such that expensive, low-value,
or long running tests are run after tests that find faults early.  We
focus on test prioritization because, at Ericsson, once a test fails an
engineer must intervene to discover the fault, so an ordering that makes the
test run fail early is more cost effective.
%Running further tests has little value. 
While early prioritization techniques continued to use coverage measures to
gauge priority, more recent approaches incorporate the faults found in past
test runs \cite{Kim2002ICSE} and change relationships among files
\cite{Sherriff2007ISSRE} to identify high value tests. 
%

We combine test prioritization techniques with empirical software engineering.
Empirical researchers use historical data to create models to help developers,
for example, identify bugs and risky changes \cite{DAmbros2010MSR}, locate
fault introducing changes \cite{Kim2006ASE}, and understand the faults found during
code review \cite{Rigby2014TOSEM}.
%and identify relevant collaborators \cite{Cataldo2006CSCW}.  
Relatively little work has combined test archives with information from other
artifacts, such as bug reports and code reviews. One of the first studies to
look at historical artifacts was done by a colleague at Concordia, Dr. Shihab,
during his PhD. In this work, he created a statistical model from the source
code history to help developers decide where to write tests for a legacy system
at Blackberry \cite{Shihab2011SPE}.

A major factor limiting research into historically based test prioritization
is information about past runs and links to other development artifacts.
Ericsson collects massive histories of logs, test runs, and development
artifacts, in part to provide for reproducibility of software and test lineups
and auditing. Much of the current analysis is done manually where developers
must sift through this information to locate the root cause of a failure. We
plan to automate this process to help in prioritization test and localizing
faults.

\section*{Research Competence} My research team has extensive training in
mining and linking software development artifacts. We have worked extensively
with semi-structured datasets that are similar to those at Ericsson and have been
able to provide useful statistical models. For example, we have mined millions
of code reviews, bug reports, tests results, and  StackOverflow discussions.
Rigby currently teaches these mining, linking, and analytics techniques in a
graduate course. Rigby has mined test data from Google Chrome in the past.
While the Ericsson data will be richer, the techniques remain similar.
The novel contribution lies in advancing our knowledge of test prioritization
and localization and in improving the efficiency of quality assurance at
Ericsson Ottawa.

\section*{Research Project and Technology Transfer}

The proposal consists of three milestones, which will run for
approximately 2 months each. First, we will investigate, clean, and link the
available data. Second we will create a test prioritization model (RQ1). Third,
we will create a fault localization model (RQ2).

{\it Investigate Process and Data:} A masters student will be onsite at Ericsson
Ottawa. His goal will be to work with Griffiths's team to understand the
quality assurance processes and to clean and link data. Ericsson’s in-kind
contribution will be substantial as they will work with the masters student and
help in interpreting the findings. I will visit Ericsson twice a month to help
the student. A PhD student, Rahman, who researches release engineering, will
also be investigating test prioritization and the impact on release
effectiveness and help generalize the master’s student findings to other
software projects.

{\it RQ1. Test prioritization model:} Not all tests are equally important, but all
tests must be run. With Ericsson's hard quality requirements, all tests must be
run before a change can be released. However, some tests can span days, while
others take seconds. Further, some tests have not failed in months, while some
are new or fail regularly. Since a failure will stop the build process and
require an engineer to intervene, a clear research direction will be to
prioritize based on past failures and test running time \cite{Kim2002ICSE,Yu2008ICSE}. A
simple ordering that makes low cost, high value tests fail early will make the
process more efficient.  Adding to these measures, we will develop change risk
and social measures, such as risk analysis based on histories of defects across
software areas and development teams. The research outcome of this stage will
be a better understanding of the archival measures that provide good test
prioritization.

{\it RQ2. Fault categorization and localization model.} Since fault localization is a
wide research area including debugging techniques and statistical bug models,
in this proposal, we focus on the narrow question -- Did that test fail because
of a problem in the development environment or product? This problem is
unusually difficult at Ericsson because the environment is exceedingly complex.
For example, test equipment is so expensive that developers in Ottawa can
request their tests be run on specialized equipment that is only available in
Sweden, and vice versa.
%
Once a test fails, a developer must determine the root cause of the failure and
whether it is an environment or product problem. Unfortunately, much of the
other information in the report is added sporadically. We will investigate the
reliability of existing fields in the reports and supplement this with freeform
data extracted using existing NLP techniques. Our goal is to train a classifier
based on past manual categorizations to help developers differentiate these
failure types. In the future, we plan to refine this coarsegrained
categorization to a finegrained localization at the file or method level by
linking issue reports to other software artifacts.

\section*{Benefits to Canada}

In 2009, Ericsson acquired the majority of Nortel's North American wireless
access business, including its market share and became the world leader in this
market. This human expertise remains in Canada where Ericsson employs about
1000 R\&D staff in Ottawa. Our contribution will be to make these individuals
more productive by improving the efficiency of the test process. If this NSERC
Engage collaboration proves to be beneficial, we will extend the project
through MITACS. We hope that this contribution leads to a long term
relationship with Ericsson Ottawa where we can attack their practical problems
and abstract them to the general advancement of knowledge in software
engineering.



\pagebreak
%\setcounter{page}{1}
\pagenumbering{gobble}
\bibliographystyle{abbrv}
\bibliography{bibliography}

\end{document}
