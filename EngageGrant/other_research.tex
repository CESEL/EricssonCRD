\documentclass[12pt, letterpaper]{article}
\usepackage{fancyhdr}
\usepackage[margin=1.87cm]{geometry}%rounded up from 1.87, just to be safe...
%\usepackage{parskip}
\usepackage{times} %make sure that the times new roman is used

\usepackage{xspace}
\newcommand{\etal}{\hbox{\emph{et al.}}\xspace}
\newcommand{\eg}{\hbox{\emph{e.g.,}}\xspace}
\newcommand{\ie}{\hbox{\emph{i.e.}}\xspace}

\usepackage{url}

\makeatletter
\renewcommand{\section}{\@startsection
{section}%		     % the name
{1}%			     % the level
{0mm}%			     % the indent
%{-\baselineskip}%	      % the before skip
{.75mm}
{.03\baselineskip}%	     % the after skip
{\normalfont\large\bf} % the style
}

\begin{document}

\fancyhead{}
%\fancyhf{}
\pagestyle{fancy}
\rhead{Test Prioritization and Localization at Ericsson -- 309207-- \textbf{Peter C Rigby}} %This puts your name at the top right, outside the margin
\renewcommand{\headrulewidth}{0pt}

\begin{center}
\begin{LARGE}
\noindent
\center{{\bf Relationship To Other Research Support}}
\end{LARGE}
\end{center}

\pagenumbering{gobble}

My research area is empirical software engineering and the goal of my ongoing
research program is to understand how and help developers to produce successful
software systems. I am also interested in statistical machine translations and
summaries of code identifiers using freeform text, such as StackOverflow
documentation.

The main goal of my NSERC Discovery grant, “Contemporary Software Peer Review:
Modern practices, fault prediction, and extraction of design decisions” is to
understand the software review practices in industrial firms as opposed to open
source projects I studied in my thesis (from this grant, I have published a
paper at the top tier conference “Foundations of Software Engineering”). 
%
In the proposed Engage grant, I will be able to use my expertise and scripts to
extract Ericsson data. A possible synergy between the grants, would be to use
peer review data in models for test prioritization. 

My Department of National Defense (DND), NSERC, industry (KDM Analytics) grant,
“The Impact of Disruptive Events on Software Systems”, involves studying
disruptive events that lead to poor software outcomes. A disruptive event, such
as developer turnover, will have risks and mitigating factors that we are
measuring. 
%
While there is no direct overlap between the grants, it may be possible to
observe how test failures disrupt the software development and release
processes.

The goal of my FRQNT grant is to translate code elements, classes and methods,
from English into French. This grant does not overlap with the Engage proposal.
%
The identifier names (eg names of classes and methods) of most major software
libraries are based on English terms that capture the purpose of identifiers
(eg the Android 'AccountManager' class). There are thousands of identifiers on
the Android project, putting non-English speakers at a disadvantage.
Statistical translations (eg Google Translate) of technical
documents using non-technical language models, results in incorrect
translations of technical terms (eg the term 'Window' has a technical meaning
quite different from its non-technical meaning). Unlike non-technical document
translation, we have observed that library identifiers are not translated when
used in multiple languages, 'AccountManager.addAccountExplicitly qu'est-ce que
cette fonction fait?' I am in a unique position to create language models to
translate software documents because, as a postdoctoral research, I developed a
technique that extracts identifier names in freeform text and code fragments.
We can take two comparable corpora, such as, the community forums on Android in
French and English and statistically determine which words tend to co-occur
with each identifier. These co-occurring words represent a language model that
describes each identifier in its respective language, which when aligned,
allows us to 1) describe the purpose and behaviour of an identifier 2) increase
the quality of translations of entire documents that discuss identifiers.


\end{document}
