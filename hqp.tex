\section*{Research Management and Training of HQP}

Students will spend 2/3rds of their time conducting research onsite at Ericsson. The remaining 1/3rd will be spend at Concordia. While at Ericcsson student will mine software test data, apply research methods to this data, and present the results through tools to Ericsson engineers. At Concordia they will generalize their findings to other projects, take required coureses, and investigate the feasibility of techniques learnt in class to advance the state of the testing art. Students may spend additional semester at Concordia finish their research theses.

There will always be two students at Ericsson and one at Concordia. Students will be managed day-to-day at Ericsson by McKenna and Griffiths. Rigby will be at Ericsson one day per week to help students with research, mining, and analytics approaches. At Concordia, students will work in our research labs.
%
Students will learn to formulate research questions, collect and clean data, create scripts to operationalize measures, store and link measures and concepts in databases, and use  data mining techniques and statistical analyses to answer our research questions. They will also operationalize their research work in the TestGuru tool.

In the table below, we describe the number and type of students that will be working on each proposed milestone. The table is followed by a description of how each research topic is interrelated allowing for synergistic work within our research groups.

%https://docs.google.com/spreadsheets/d/1o7cTGi406Mc8Qx4KWPOByfPNKhB4lydCUQb_G4ngzrE/edit#gid=0
%convert to latex: http://ericwood.org/excel2latex/
\begin{table}[h]
\center
\caption{Allocation of students for each proposed milestone. Rehman and Amar are masters students who have been partially funded by an NSERC Engage and MITACS. We will hire a new PhD student and two new masters students.}
\label{tableHQP}
\vspace{+3mm}

\begin{tabular}{  l | c | c | c  }
\hline
	\textbf{Year} & 1 & 2 & 3 \\ \hline
	1. Test Effectiveness, Test Prioritization, and Test Groups & PhD & PhD & PhD \\ 
	2. Anomalies: Environment vs Product & Rehman &  & \  \\ 
	3. Fault Localization through Test Logs & Amar & Amar (.5) & \  \\ 
	4. TestGuru: Risk and History & & Masters (1.5) &  Masters (2)  \  \\ \hline
	Total students & 3 & 3 & 3 \\ \hline

\end{tabular}

\end{table}

%\vspace{-.5mm}

We plan to recriut a PhD student to work on the first milestone which inovlves test simulations of test prioritzation and determining dependent tests. Although there is a wide range of work in this area, the students research will be conducted in a more applied setting.

Through our NSERC Engage and MITACS our masters student Rehman has conducted begun work on Milestone 2. He has determined the failure distributions for Ericsson test cases and the report he creates is used by Ericsson developers. He is returning to Concordia to generalize his results. Herzig at Microsoft will be providing anonymized test failure distributions, so that he can determine the degree of anomaly or flakiness at microsoft. He will also be replicating his work on open source projects include Chrome, which is developed in Montreal. He will also be working to integrate his work into TestGuru. He will finish his Masters thesis at the end of the first year of the proposed grant.

Supported by a MITACS, our masters student Amar began work on Milestone 3: fault localization through test logs in February. He has extracted the necessary logs and has begun to iterate over the steps outlined in Section~\ref{secLogs}. He will be supported by the proposed grant for just under two years.

Milestone 4 will start in year two of the grant as it takes advantage of the data extracted and the lessons learned from the previous milestones. We will hire two masters students to agment this data with other historical data. One student will use this data to suggest the areas of the system most at risk that require additional tests to be created. The other will work with the PhD student to the agmented data to perform test prioritization. Both students will be responsible for working with Ericsson testers to conduct usability studies of their tools and to integrate them into the Ericsson development process.

