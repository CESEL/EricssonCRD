\documentclass[12pt, letterpaper]{article}
\usepackage{fancyhdr}
\usepackage[margin=1.87cm]{geometry}%rounded up from 1.87, just to be safe...
%\usepackage{parskip}
\usepackage{times} %make sure that the times new roman is used
\usepackage{comment}

\usepackage{xspace}
\newcommand{\etal}{\hbox{\emph{et al.}}\xspace}
\newcommand{\eg}{\hbox{\emph{e.g.,}}\xspace}
\newcommand{\ie}{\hbox{\emph{i.e.}}\xspace}

\usepackage{xcolor,colortbl}
\newcommand{\todo}[1]{\textcolor{red}{TODO: #1}\PackageWarning{TODO:}{#1!}}

\usepackage{url}

\makeatletter
\renewcommand{\section}{\@startsection
{section}%		     % the name
{1}%			     % the level
{0mm}%			     % the indent
%{-\baselineskip}%	      % the before skip
{.75mm}
{.03\baselineskip}%	     % the after skip
{\normalfont\large\bf} % the style
}

\begin{document}

\fancyhead{}
%\fancyhf{}
\pagestyle{fancy}
\rhead{-- 309207-- \textbf{Peter C Rigby}} %This puts your name at the top right, outside the margin
\renewcommand{\headrulewidth}{0pt}

\begin{center}
\begin{LARGE}
\noindent
\center{{\bf Test Effectiveness, Localization, and Prioritized Generation in Complex Test Environments}}
\end{LARGE}
\end{center}


%sections: http://www.nserc-crsng.gc.ca/OnlineServices-ServicesEnLigne/instructions/101/crd_eng.asp
%Synopsis

A cellular base station connects cell phones to a voice and data network via
LTE (4G) or 3G standards. The software that runs on these base stations
contains not only complex signalling logic with stringent real-time
constraints, but also must be highly reliable, providing safety critical
services, such as 911 calling. At Ericsson Ottawa, testing base station
software is time consuming and involves expensive specialized hardware. For
example, testers may need to simulate cellular devices, such as when a base
station is overwhelmed by requests from cell users at a music concert. In
order to maximize the return on investment of Ericsson's testing efforts, 
we plan to use historical artifacts including past code changes, bugs, peer
reviews, test logs, and test runs to create statistical models (1) to determine effective environmentally valid test priortizations, (2) to differentiate between environment and product test failues, (3) to locate faults by cleaning the noise out of test logs, (4) to quantify change risk and identify software units that need new tests, and (5) to integrate these actionable results into our proposed tool called TestGuru. The outcomes of this work will advance the state-of-the-art
in test prioritization, test log analysis, and fault localization and provide
Ericsson with a more efficient test infrastructure.

\section*{Background: Overview of Research Approach and Data}

The proposed work intersects three areas of software engineering: test
prioritization and effectiveness, test log analysis, and empirical software engineering. 

Regression testing research has three streams of research~\cite{Yoo2012STVR}.
%
%: minimization, selection, and priorization\cite{Yoo2012STVR}.
%
The first, {\it minimization}, involves eliminating tests that are redundant or
of low value. In the literature, the problem has been reduced to one of code
coverage, for example, tests become redundant as the system evolves and more
than one test covers the same control flow. As a result, much of the work in
this area is algorithmic, such as transforming it into a spanning set problem
\cite{Marre2003TSE}, using divide-and-conquer strategies \cite{Chen1996IPL},
and greedy algorithms \cite{Tallam2005SENotes}. 
%
%These algorithms are based on source code coverage, and miss important
%historical information, such as whether a file is at higher risk to failure
%because it has been changed recently.
The second, {\it selection}, uses the same static analysis techniques such as
coverage \cite{Taha1989COMPSAC} and slicing \cite{Jeffrey2006COMPSAC}, but
selects tests that cover source files that are at higher risk because they have
been changed recently \cite{Rothermel1994ICSE}. 
%
The third, {\it prioritization}, orders tests such that expensive, low-value,
or long running tests are run after tests that find faults early.  We
focus on test prioritization because, at Ericsson, once a test fails an
engineer must intervene to discover the fault, so an ordering that makes the
test run fail early is more cost effective.
%Running further tests has little value. 
While early prioritization techniques continued to use coverage measures to
gauge priority, more recent approaches incorporate the faults found in past
test runs \cite{Kim2002ICSE} and change relationships among files
\cite{Sherriff2007ISSRE} to identify high value tests. 

We combine test prioritization techniques with empirical software engineering.
Empirical researchers use historical data to create models to help developers,
for example, identify bugs and risky changes \cite{DAmbros2010MSR}, locate
fault introducing changes \cite{Kim2006ASE}, and understand the faults found
during code review \cite{Rigby2014TOSEM}.
%and identify relevant collaborators \cite{Cataldo2006CSCW}.  
Relatively little work has combined test archives with information from other
artifacts, such as bug reports and code
reviews~\cite{Shihab2011SPE,Herzig2014ISSRE}. One of the first studies to look
at historical artifacts was done by co-investigator Dr. Shihab, during his PhD.
In this work, he created a statistical model from the source code history to
help developers decide where to write tests for a legacy system at Blackberry
\cite{Shihab2011SPE}.

A major factor limiting research into historically based test prioritization is
information about past runs and links to other development artifacts.  Ericsson
collects massive histories of test logs, test runs, and development artifacts, in
part to provide for reproducibility of software and test lineups and auditing.
Much of the current analysis is done manually where developers must sift
through this information to locate the root cause of a failure. We have begun
to automate and guide developers to help with prioritization test and
localizing faults.





\section{Research Milestones} 

This research proposal builds upon an NSERC Engage and an ongoing MITACs. For
each milestone, we describe the work that we have done over the last six months
and the direction we plan for this proposal. The proposal consists of \todo{X}
milestones. In Milestone 1, we will evaluate test effectiveness and build a
test prioritization model. In Milestone 2, we will use the test failure
distribution to find anomalies and differentiate between product and
environmental faults. In Milestone 3, we will improve fault localization by
mining test logs. In Milestone 4, we integrate our results and use them to
quantify the change risk associated with test failures and relate them to
customer found defects to identify which tests should be run for a change,
which tests should be run after a failure, and which tests should be written to
stop fault slipthrough to customers.  We integrate our work into a framework
that provides actionable, timely results to Ericsson developers, we call this
tool TestGuru.

%peter needs to re-work parts of this
\subsection{Test Effectiveness, Test Prioritization, and Test Groups}

%Ericsson problem: overwhelmed by size
The complex test infrastructure and massive number of tests that run each night makes it difficult for Ericsson testers and managers to determine how effectively they are testing their software. We calculate test effectiveness and develop an initial test prioritization scheme. We plan novel work on prioritization in the context were not all tests are independent.

%We began our NSERC Engage by dd
We began our NSERC Engage by extracting, cleaning, and linking test runs with problem reports and the version control system. Ericsson managers wanted to understand two basic questions: which software units (areas of the system) have the largest number of failing tests, and which tests find the largest number of faults. 
%
This initial work took a researcher associate, Zhu, four months. The industrial outcome of this work has been significant. We found that the top 25\% of tests, ranked by past faults detected, found over 80\% of future faults.
%
%Ericsson improved its nightly test performance by running the tests that found few faults less frequently. \todo{I think this has had problems because really they are KPIs. Add Maaz's results? Also, what about the fact that its hard to reorder} 
%
This helped test managers in their ongoing test re-writing and re-organization efforts. Further, the development team received a list of their most common software units which lead to discussion of the cause of the failures allowing them to mark code and tests for re-work. 

From a research perspective, our results replicate findings in a new context. As Mockus \etal~\cite{Mockus2009ESEM} found, having high test coverage comes at the cost of high redundancy. This redundancy results in high maintenance overhead as developers write tests that are not necessarily effective at finding customer defects. Proposed work on this finding is discussed in Section~\ref{secTestGuru}. Our findings also confirm the growing consensus that tests can be effectively prioritized by how often a test has failed in the past~\cite{Kim2002ICSE,Hemmati}. In this proposal, we constrain possible prioritization or reordering into dependent groups of tests to reduce the false positives introduced by inappropriate reordering.

Existing work on test prioritization usually assume that tests are independent and can be run in any order~\cite{Marijan2013ICSM,Elbaum2014FSE,Hemmati2015ICVV}. In test environments that involve complex hardware and software, the order of test cannot be assumed to be independent. When teams at Ericsson in Sweden reordered their test, many test began to fail. A large and expensive effort to make tests was undertaken. The goal of this milestone is to determine the value of reordering tests and determine which tests are dependent.

\textbf{Dependent Groups of Tests}\\
We propose a novel research approach that groups sets of dependent tests into atomic units that can be reordered only as as a group. 
%
First, we plan to run simulations on past test runs to show that the new ordering would have found faults earlier and with less test runs. The simulation is evaluated based how much earlier a fault is found and how many fewer tests need to be run to find the fault. 
%
In the simulation we use the past outcome and no new tests are run. This requires us to assume that tests are independent and can be ordered arbitrarily. If the simulations show that the benefit of re-ordering is low, then the product team need will not need expend any further effort and can keep the existing order.

Second, to remove the test independence assumption, we must run actual tests with a new order. At Ericsson tests are expensive and we cannot re-run the entire suite of tests. 
%
Since we must conduct our experiment of test ordering in a live environment, there are two possibilities when a failure occurs: the test order is not valid or a fault has been found in the system. To deal with these two conditions, we design our experiment as follows:

\begin{enumerate}

\item Prioritize tests on the basis of their past test failure distribution~\cite{Kim2002ICSE,Hemmati}. We also plan to use other historical information, such as author expertise and file churn, see Section~\ref{secTestGuru}.

\item If a group of tests fail, re-run in the normal order. If there are no fails on re-run, then we mark the tests as dependent. If there are fails on re-run, we send the failures to Ericsson testers for investigation because it is likely true failure. Since failing tests are automatically re-run at Ericsson, the additional cost will be minimal.
 
\item If the tests pass, we mark them as independent.

\end{enumerate}

Initially, there will be large number of failures that will be the result of inappropriate orderings. However, as our dataset grows, we will be able to determine the likelihood that a failure is related to a defect vs a failure related to an inappropriate ordering. Tests that fail together more often than by chance will be grouped into atomic units. These groups of tests will be reordered as a single grouping.

The final stage of this research will be to simulate the added benefit of breaking a test group. We will simulate the time savings and compare it with the test dependencies to provide a list of test groups ranked by a cost benefit analysis. This simulation will allow Ericsson managers to determine which test groups to re-structure first.

%Novel Future: using co-failure distributions to order tests, Ans: which test will give the most new information, add to TestGuru




%a bit of redundancy here
\subsection{Anomalies: Environment vs Product}

Fault localization is a wide research area including debugging techniques and statistical bug models, so in the NSERC Engage work, we focused on the narrow question -- Did that test fail because of a problem in the development environment or product? This problem is unusually difficult at Ericsson because the environment is complex. As a result, there are \todo{eight} Ericsson Ottawa developers who examine the failures that occur in the system after the nightly test run. Our goal in this milestone to help the team differentiate between environment and product faults and to prioritize the failing tests for investigation.

The expected novel contributions in this milestone are twofold. First, most prioritization techniques determine the order of tests to be run, in contrast, we determine the order of tests to be investigated after a failed run. Second, to categorize faults as product or environment, we conduct an anomaly analysis. Our list of anomalies is used each morning at the testing team's 10am meeting to determine which faults each developer will investigate. 

Is a test failure anomalous? The nature of base station tests makes Ericsson's test environment complex and as a result many tests fail sporadically from environmental problems. Our work has focused on determining the test failure distribution for each test. We label tests that fall outside a 95\% confidence interval to be anomalies. Each test has its own failure distribution. For example, tests that fail regularly because of environmental problems will have to fail at a statistically significant higher level to be flagged as a potentially interesting test failure. In contrast, a test that passes every day for two months and fails on a nightly run, will be flagged as an anomaly. 

The outcome of this work has been very positively received at Ericsson Ottawa. For example, the testing team, which consists of 8 developers who runthrough test failures every day, use the test anomalies as a technique to prioritize failing tests for investigation. Our interactions with these testers has lead to additional enhancements, such as determining the number of times a test has been an anomaly on the scale of a week, month, and for the current release. Proposed work involves conducting user studies of these testers to determine what and how to show them our results and integrate results into TestGuru. 
%Future: we continue to the add datapoints to the TestGuru reports as we get feedback from the test teams. We are also experimenting with co-failure distribution and clustering. We further dicuss this in \todo{Section} 

From a research perspective it is important to understand if this technique can be applied in other settings. We plan to analyze the failure distribution from other Ericsson sites in Sweden. At our suggestion, Herzig, at Microsoft Research is running the anomaly detection technique at Microsoft. His findings indicate that anomalous test failures are almost always environmental faults and that can be safely ignored as they will not usually reappear in a subsequent run.




%logs and test for Ian
\subsection{Fault Localization through Test Logs}

Identifying underlying fault of a test failure is challenging. Unlike conducting code review, where developers manually read over the code to directly identify issues~\cite{}, a test failure can be a symptom of multiple underlying faults and one fault may lead to multiple test failures. Therefore, the diagnosis of test failures in to locate underlying faults can be time consuming~\cite{}. This is especially true at Ericsson where a test may fail one day due to an environment problem and another day from a product fault. Although our anomaly analysis helps prioritize the test failures and gives a sense of how long a test has been failing, it does not help provide a precise reason and location for a fault
%, e.g., whether the test is failed due to a newly introduced fault. 
To assist practitioners of Ericsson in locating test faults, we plan to conduct log analysis on tests. Our co-investigator Dr. Shang is an expert in log analysis. We currently have a masters student, Amar, who is partially funded by a MITACs and has started to collect and clean the Ericsson logs.

Our preliminary analysis of test logs at Ericsson has revealed three challenges in: 1) there is an overwhelming amount of detailed log information, 2) the complex testing environments and context information brings noise into the logs and 3) the complex testing scenarios brings additional variations and noise into logs. We plan our approach around these challenges. 

\begin{enumerate}

\item Data collection: We plan to extract all available the test data including a stream of events, counters, and logs. In order to assist in understanding the complex testing environments, we extract the units from the data, \ie which test basestation nodes are connected to which test mobile devices.

\item Log abstraction: We plan to abstract logs in order to reduce the contextal noise. Logs contain static and dynamic information. The static information is specific to each particular event, while the dynamic information describes the event context. We plan to abstract logs into events by vocabulary analysis \todo{cite}. The words that exist in the vocabulary of logs but not in the vocabulary of source code are considered dynamic values. By anonymizing such dynamic values, we can abstract logs in to events without context noise. 

\item Event sequence generation: The large volume of logs makes it difficult
and time consuming to understand each of the individual log lines. We plan to
generate event sequences in order to illustrate the execution flow of the
tests. The complex testing scenarios increase the variability of events in a
similar execution flows. We compress event sequences based on testing units to
extract generalized event sequences and to remove irrelevant, redundant, and
highly variable event sequences.

\item Event sequence comparison: We compare event sequences and calculcate the delta of event sequence to localize test faults. Our comparison consists of two folds: 1) we compare event sequences from runs that are successful with those that are unsuccessful to identify potential problematic events and 2) we compare event sequences from current tests and prior tests to identify whether the tests are failed due to different faults.

\item Fault localization: Based on the identified potential problematic event, we perform three types of fault localization. 
	\begin{itemize}
		\item Environmental fault localization. We examine the identified problematic events in different testing environments. For example, we calculate the probability of observing the problematic events for different testing nodes and different mobile devices. If the events appear frequently in a special type of testing node or mobile devices, we locate the fault for such node and devices. 
		
		\item Source code fault localization. With the problematic events identified from last step, we are able to find the source code that generate such log events. Dr. Shang has proposed a technique to link log lines to source code~\cite{Shang:2014:ULL:2705615.2706065}. We first leverage Dr. Shang's approach to first locate the source code where the problematic event is generated. However, there often exist high complexity in the course code of large systems, especially for the system in Ericsson. We want to recovered a more detailed context (control flow and data flow) of the systems when the problematic event is generated. In order to recover such context, we plan to first map the dynamic values in the log lines with the variables in the source code. For example, by observing a log line as ``Error, retry$=$10'', we would like to link the number ``10'' to a variable in the source, such as ``int retryNumber''. By data flow analysis on the variable, we can have rich knowledge about the context of the execution. In addition, we recover a control flow of the problematic event. By analyzing the control flow, we can have the knowledge of the values of condition statements that may lead to the problematic event. We do not only recover such data and control flow of the problematic event, but also for all the events in the event sequence that is generated from last step. We use such information to automatically location faults in the system. Such information can assists developers to future diagnose the faults.
		
		\item Historical fault localization. Many faults are repetitive and can be seen or resolved before. For each identified and resolved faults, we attached problematic events and the event sequences with the fix of the fault. When a problematic event is identified, we match the problematic event and event sequences with the resolved faults. If we find similar event sequence and problematic events that are attached with priorly resolved faults, we suggest developers to examine the prior fault before digging into the current fault. In addition, the people who resolved prior faults can be ideal candidates as experts to fix the current fault.
		
	\end{itemize}
\end{enumerate}

We plan two evaluation plans to measure the success of our approach. 
\noindent \textbf{Localizing injected faults.} First, we plan to evaluate our approaching by injecting faults into the system. We inject common faults (including environmental faults and functional faults) that are priorly identified by practitioners into the system. Then we leverage our approach to perform fault localization. For the environmental faults, we examine whether we can identify the particular environment that has the fault. For the functional faults, we examine whether the context of the source code that is localized by our approach can unveil root cause of the fault. For every injected fault, we examine the prior fault that is located by our historical fault localization to understand whether the historical knowledge can assist in diagnosing the fault. To have a comprehensive evaluation of our approach, we plan to inject faults into systems in Ericsson as well as open source system software, such as Hadoop.

\noindent \textbf{Evaluating by deploying our approach into practice.} Second, we plan to evaluate our approach with the practitioners in Ericsson. We deploy our approach in the real testing environment in Ericsson. For every fault from the testing environment, we provide our fault localization, and we record the actual fix of the fault. We compare the actual fix of the fault with our suggested fault localization. We want to examine: 1) how often does our approach gives accurate fault localization and 2) whether the information provided by our approach would help speed up the diagnose and fix of the fault.



% oops should have used this list:
%1. Abstracting logs into a sequence of events to illustrate the execution flow of the tests.
%2. Compare event sequences to determine whether the failed test are due to new root causes or existing ones.
%3. Linking logs to codes to identify/scope down? the scenarios (root causes) of the failed test.

Our first novel contribution is to combine both run-time system behaviour during test (i.e., logs) and test results (i.e., pass or failure of the test) to provide actionable suggestions, such as fault localization, to practitioners. Similar approaches have been successful applied by Shang~\cite{Shang:2013:ADB:2486788.2486842} to compare the logs from test and production environments. In Ericsson's context, we do not have the production logs, so we must change our comparisons to involve previous test logs from other releases and successful and unsuccessful runs. Secondly, we consider the run-time system behaviour during test as a historical repository. Such system behaviour consists of rich information of the system during execution. We believe the knowledge from learning and analyzing the historical behaviour of the system are of huge benefits to software development and operation.

%\todo{Ian: further novel aspects include ...}


%testGuru for Emad
\subsection{Milestone 4, TestGuru: Risk and History}
\label{secTestGuru}

Our work with Ericsson has taken applied problems and developed research-based solutions. To ensure that our results are acted upon, we plan to integrate our results into TestGuru. TestGuru will be a web-based tool that consists of three components: determining risky changes, test prioritization, and test creation. We elaborate on each part below.

\textbf{Leveraging test history to determine risky changes:} Shihab~\cite{Rosen2015FSE} has developed a tool, called CommitGuru, that feeds factors extracted from the software changes and files that are part of the change to determine the likelihood of a change introducing a future defect \ie the riskiness of the change. CommitGuru uses a logistic regression model, that is trained on historical data, to calculate the probabilities of risk for each change. The model features include the number of source code lines changes, the expertise of the author, and the diffusion/dispersion of the change. A major drawback of the current implementation of CommitGuru is that it only uses traditional code (e.g., Cyclomatic complexity) and process (e.g., the number of prior changes) metrics to determine the risk of changes, and does not help with the localization within changes. 

As part of the this proposal, our goal is to add historical test executions and test log information to assess the risk of a change. While some preliminary works have investigated adding these features to statistical fault model~\cite{Herzig2014ISSRE}, we will conduct a systematic empirical study of multiple projects to determine how effective the use of historical test data is at predicting risky changes. The features extracted from the test data will mainly focus on leveraging the history and the complexity of the tests. For example, if a test rarely fails, it may be more important than a test that always fails. We will validate our change risk prediction through precision and recall. We will also validate the accuracy of the techniques with Ericsson developers to make sure that our techniques are practical. For example, since Ericsson developer investigate a large number of failing test each day, a technique with low precision but high recall may be appropriate in practice.

\textbf{Test prioritization:} Whereas our first component focuses on test history to determine which changes are risky, our second goal of TestGuru is to help prioritize tests based on their effectiveness. Much of the test prioritization work has focused on the organization of tests based on coverage (e.g.,~\cite{AggrawalSEN04}).  Shihab et al. have leveraged the development history to prioritize the creation of tests. Similarly, we would like to use the history of how effective a test was in findings customer defects (which we call \emph{historical effectiveness of the test}) to prioritize current tests (rather than prioritize the creation of tests). In addition to using the historical effectiveness of the test, we plan to leverage the \emph{co-failure information of the tests} to dynamically order tests. For example, if test $B$ always fails when test $A$ fails, then we can re-order our tests and push back the execution of running test $B$ if we see that tests $A$ failed. Such a strategy will allow us to execute the most diverse tests early on so that failures can be found earlier. Prior work on random testing \cite{Duran84TSE, Arcuri2012TSE} proposed the idea of running tests in a diverse way, however, the our idea is to use historical knowledge, which in this case is the co-failure distribution of tests, to guide the prioritization. 

We plan to validate the effectiveness of the proposed test prioritization using Ericsson's testing data. In particular, we plan run simulations based on the historical test data gathered by Ericsson to determine 1) how many actual defects can we detect using the proposed strategies and 2) how quickly can we detect these tests. We will compare our two strategies with the current strategies used by Ericsson's testing teams to determine the amount of improvement, if any, of the proposed strategies. To draw more general conclusions, we plan to replicate our study on a large set of open source projects and contrast the open source and Ericsson findings.

%Instead of predicting change risk, we can use the same predictors to determine which are the most effective tests. This will build on our first milestone that is limited to co-failure distribution and add information such as developer expertise. \todo{Could make this more interesting by suggesting the next test to be run instead of prioritizing upfront} Test re-ordering 

\textbf{Test creation:} Ericsson managers provided use with examples where there are many tests execution that pass, but customers still find defects. Hence, in addition to the prioritization of existing tests, new tests always need to be written. In practice, writing tests for all of the new code is infeasible and an ineffective use of resources~\cite{Mockus2009ESEM}. We plan to combine our change risk models, along with the test effectiveness approach to determine areas of the code that are risky and under tested, \ie need new tests to be written or have current tests re-written. In particular, once our models indicate that a change is risky, we will look into the tests that are written for the code in that change. We will also examine the historical effectiveness of the existing test and determine whether new tests need to be written or existing tests need to be re-written. To accomplish this, we plan to flag areas of the code that need tests to be created, however, the actual creation of the tests will need to be performed by the developers or testing teams who have domain knowledge.

To validate our approach, we plan to run a historical simulation and to examine the effectiveness of the written tests after a delay period, of for example, 1 year. In the simulation, we suggest the code units that developer should write new tests for. We then determine, how many defects these code units have in the future. If we suggest code units that are buggy our strategy is effective. To complement the simulations, we plan to also examine the effectiveness of any tests created as a result of our proposed test creation strategy 1 year of after each test's creation. Such an experiment is critical since it will allow us to gain insights into what works and what does not, as well as, allow us to ask the creators of the tests their rational for the test's creation, why they think it was effective, and how we can enhance our proposed strategy. This feedback demonstrates the value of the research partnership with Ericsson. 

%Our models will be helpful in indicating the areas of the system that need critically need new test or need tests to be re-written.




\section*{Team Expertise}

\section*{Research Management}

\section*{Training of HQP}

\todo{student names and schedule of 8 monts Ericsson, 4 at concordia}

\section*{Research value and industrial relevence}

\todo{put outcomes in here}

\section*{Benefit to Canada}

In 2009, Ericsson acquired the majority of Nortel's North American wireless
access business, including its market share and became the world leader in this
market. This human expertise remains in Canada where Ericsson employs about
1000 R\&D staff in Ottawa. Our contribution will be to make these individuals
more productive by improving the efficiency of the test process. If this NSERC
Engage collaboration proves to be beneficial, we will extend the project
through MITACS. We hope that this contribution leads to a long term
relationship with Ericsson Ottawa where we can attack their practical problems
and abstract them to the general advancement of knowledge in software
engineering.


\pagebreak
%\setcounter{page}{1}
\pagenumbering{gobble}
\bibliographystyle{abbrv}
\bibliography{bibliography}

\end{document}
