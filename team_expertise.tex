\section*{Team Expertise}

The team is composed of researchers, the students in their labs, and testers and test managers at Ericsson. Below we describe how each member's skillset contributes to the proposed applied research proposal.

Shihab has extensive experience mining software repositories and presenting the results in an actionable form.
%~\cite{Shihab2011SPE}
For example, collaborating with Blackberry he created a statistical model from the source code history to help developers decide where to write tests for a legacy system. He developed and maintains the CommitGuru tool that uses statistical bug models to assign risks to commits. Combining testing and modelling we plan to make our test research actionable in the TestGuru tool.

Shang's PhD and current research program focuses on mining logs to find bugs and performance issues in production systems.
%~\cite{Shang:2013:ADB:2486788.2486842}. 
In this proposal, Shang will lead the mining of test logs to help in fault localization.

Rigby's research lab has extensive training in mining and linking software
development artifacts. They have worked extensively with semi-structured
datasets that are similar to those at Ericsson and have been able to provide
useful statistical models. For example, they have mined millions of code
reviews, bug reports, tests results, and  StackOverflow discussions. Rigby has
mined test data from Google Chrome and gave a keynote at the
Microsoft testing workshop on his work to identify flaky tests.  While the
Ericsson data will be richer, the techniques remain similar.  The novel
contribution lies in advancing our knowledge of test prioritization and
localization and in improving the efficiency of quality assurance at Ericsson
Ottawa.


The main contacts at Ericsson Ottawa are experienced software managers, testers, and developers. Griffiths is the Section Manager for the Development
Environment in Ottawa. He has extensive knowledge of the environmental issues, tools, and techniques that are being successful used at Ericsson. McKenna is the lead
test Engineer at Ericsson Ottawa. He has a pragmatic approach to ensuring that Ericsson's continuous integration environment runs smoothly. Both Griffiths and McKenna are always available to discuss issues with onsite students. At the biweekly meetings of the team, they provide useful feedback and help ensure that the research remains useful to the test teams. As a result, the testing group lead by McKenna uses our daily anomaly analysis report to prioritize failures to investigate in their morning meeting.
